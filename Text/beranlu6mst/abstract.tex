\startAbstractCz
  Dynamické vyvažování obtížnosti slouží k lepšímu přizpůsobování se programů úrovni uživatelů. Pro tuto oblast existuje mnoho různých přístupů a jejich aplikací. Jednou z aplikací je dynamické vyvažování počítačových her, kde je jejím úkolem udělat hru zábavnější. Jednotlivé přístupy mají své metriky, jak zábavu měřit, a to využívají při vyvažování.
	
V teoretické části práce shrnuji existující přístupy a metriky zábavnosti. Dále navrhuji nový přístup založený na hráči prostředí. Tento hráč ovlivňuje náhodu a skrytou informaci ve hře za účelem lepšího požitku z ní. Pro účely nového přístupy definuji metriky nové.

V praktické části popisuji testující prostředí s implementovanými hrami Bludiště, Ludo a Ztracená města, které jsem použil pro evaluaci algoritmů hráče prostředí a pro porovnání s dvěma existujícími algoritmy.
\stopAbstractCz

\startAbstractEn
Neaktuální.
  Dynamic difficulty adjustement serves for better software adaptivity to user skills. In theoretical part I summed up existing approaches and applications. I define fun and I suggest metrics of fun. I suggest new approach based on environmental player. This player control chance and hidden information in game with purpose of better fun. In practical part I describe testbed with games Maze, Ludo and Lost Cities which I have used for evaluation of environmental player algorithms and for compering them with two existing approaches.
\stopAbstractEn

\endinput
%%
%% End of file `abstract.tex'.
