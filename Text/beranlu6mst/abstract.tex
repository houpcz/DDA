\startAbstractCz
  Dynamické vyvažování obtížnosti slouží k lepšímu přizpůsobování se programů úrovni uživatelů. Pro tuto oblast existuje mnoho různých přístupů a jejich aplikací. Jednou z aplikací je dynamické vyvažování počítačových her, kde je jejím úkolem udělat hru zábavnější. Jednotlivé přístupy mají své metriky, jak zábavu měřit, a to využívají při vyvažování.
	
V teoretické části práce shrnuji existující přístupy a metriky zábavnosti. Dále navrhuji nový přístup založený na hráči prostředí. Tento hráč ovlivňuje náhodu a skrytou informaci ve hře za účelem lepšího požitku z ní. Pro účely nového přístupy definuji metriky nové.

V praktické části popisuji testující prostředí s implementovanými hrami Bludiště, Ludo a Ztracená města, které jsem použil pro evaluaci algoritmů hráče prostředí a pro porovnání s dvěma existujícími algoritmy.
\stopAbstractCz

\startAbstractEn
Dynamic difficulty adjustment serves for a better software adaptability to user skills. For this area there are many different approaches and their applications. One application is a dynamic difficulty adjustment in computer games where its goal is to make the game more fun. All approaches have their own metrics how to measure fun. They are using it for a game balancing.

In theoretical part I summed up existing approaches and fun metrics. I suggest new approach based on environment player. This player controls a chance and a hidden information in the game with purpose of better fun.

In practical part I describe test environment with games Maze, Ludo and Lost Cities which I have used for evaluation of environment player algorithms and for comparing them with two existing approaches. 
\stopAbstractEn

\endinput
%%
%% End of file `abstract.tex'.
