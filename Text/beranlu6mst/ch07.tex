\chapter{Závěr}

Cílem diplomové práce bylo seznámit se s problematikou dynamického vyvažování obtížnosti v počítačových hrách především z hlediska algoritmů teorie her, zanalyzovat existující přístupy a metriky pro měření úspěšnosti jednotlivých přístupů, navrhnout nový přístup a metriky. Dalším cílem bylo navrhnout a implementovat testovací prostředí s třemi hrami a na nich nový přístup ohodnotit dle existujících i nových metrik. 

Nalezené existující přístupy využívají velké množství metod umělé inteligence. Jsou zde zastoupeny neuronové sítě, POMDP, fuzzy logika, evoluční algoritmy a další. DDA v oblasti teorie her se ukázalo být zatím nedostatečně prozkoumáno, a proto by tato práce měla částečně zaplnit tuto mezeru.

Navrhl jsem algoritmy založené na hráči prostředí, který má za úkol přizpůsobovat náhodné události a skrytou informaci. Při používání DDA je jedním z hlavních rizik hráčovo zjištění, že je DDA ve hře použito. Z tohoto důvodu bylo zásadní navrhnout algoritmy tak, aby upravovaly náhodné události v uvěřitelné míře. Pro tento účel se ukázala čtveřice existujících metrik zábavnosti - poměr vítězství, změna vedení, náskok a napětí - nedostatečná. Z tohoto důvodu jsem postupně dodefinoval další pětici metrik - doba vedení, uvěřitelnost, spravedlnost, svoboda a náhodnost, která existující metriky vhodně doplnila. Algoritmy založené na hráči prostředí jsem nazval E-HillClimber, E-MaxMax, E-Max$^n$ a E-MonteCarlo dle známých algoritmů z teorie her, z kterých vycházejí.

Pro ověření funkčnosti nových algoritmů jsem navrhl a implementoval testující prostředí s třemi hrami - Bludiště, Ludo a Ztracená města. Aplikace obsahuje režim pro hraní hry člověkem proti soupeřům a režim dávkového spouštění experimentů na více vláknech.

Každý z experimentů jsem nejdříve spustil bez jakéhokoli mechanismu DDA, poté se čtveřicí algoritmů hráče prostředí a nakonec s dvěma existujícími algoritmy dynamická úroveň a POSM pro porovnání. Výsledky experimentů jsou pozitivní. Hráč prostředí se ukázal jako vhodným konceptem, který předčil hru bez žádného DDA mechanismu i algoritmy dynamickou úroveň a POSM ve všech třech testovaných hrách. 

V experimentech nejlépe dopadl E-Max$^n$, který v některých případech dokázal předčit všechny ostatní algoritmy dle většiny metrik. E-HillClimber dosáhl o něco slabších výsledků, ale stále oproti existujícím přístupům i proti nevyužití žádného DDA si vedl velmi dobře. E-HillClimber je jednoduchým algoritmem, který je díky svým malým požadavkům a velké rychlosti použitelný nejen v tahových hrách.

Do budoucna lze navázat na výstup této diplomové práce především otestováním algoritmů na lidech, důkladným zkoumáním vlivu změn koeficientů metrik u jednotlivých her, případně s návrhem jiné maximalizační funkce než je vážená suma metrik.

Diplomovou práci v této podobě považuji za kompletní. Věřím, že plně splnila zadání.

\endinput
%%
%% End of file `ch01.tex'.
