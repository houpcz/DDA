\chapter{Testující prostředí}

\section{Použité hry}

Pro testování navržených algoritmů v předchozí kapitole jsem si zvolil následující tři relativně jednoduché hry - Bludiště, Ludo a Ztracená města. Hry se liší počtem hráčů, úplností informací a determinismus. Zařazení do jednotlivých kategorií znázorňuje následující tabulka \ref{tab:klasifikaceher}.

\begin{table*}[b]\footnotesize
\vspace*{0mm}
\caption{{\label{tab:klasifikaceher}} Klasifikace her do různých tříd dle počtu hráčů, determinismu a úplnosti informace. }
\vspace*{0mm}
\label{shadowtable}
\begin{center}
\begin{tabular}{| l || c | c | c |}
\hline
Název hry & Počet hráčů & Determinismus & Informace \\
\hline
\hline
Bludiště & 1 & Bez náhody (z pohledu hráče) & Neúplná \\ \hline
Ludo & 4 & S náhodou & Úplná \\ \hline
Ztracená města & 2 & S náhodou & Neúplná \\ \hline
\end{tabular}
\end{center}
\end{table*}

Schéma podsekcí jednotlivých her je stejné. Nejdříve popíši cíl a pravidla hry. Následuje popis hráče prostředí (HP) v dané hře. Nakonec popisuji funkci hodnosti, status funkci a výpočet uvěřitelnosti, které jsou využívány metrikami pro měření zábavnosti. Pojmy funkcí hodnosti a statusu vychází z algoritmu Dynamická úroveň (\ref{sec-dynlevel}).

\subsection{Bludiště}

Bludiště je jednoduchou hrou pro jednoho hráče. Na mapě je umístěn neznámý počet bomb. Cílem hráče je tyto bomby včas najít a zneškodnit.

Herní plán bludiště se skládá z 2D čtvercové sítě. Jednotlivé čtverce mapy představují zdi, dveře, bomby, nebo průchozí pole. Hráč vidí bludiště z ptačí perspektivy a bludiště objevuje postupně. Vždy po otevření dveří se mu zobrazí chodba za nimi. Všechny bomby vybuchnou ve stejný čas. Čas je zde měřen na počet kroků hráče.

Terorista navíc na některé dveře připevnil senzory, a poté dveře přemaloval namodro, nebo červeně. Jestliže hráč otevře modré dveře, získá čas navíc. Otevře-li červené, čas se mu zkrátí.

V prostředí jsou implementovány dvě varianty hry. V první hráč od začátku zná pozici všech bomb, musí pouze nalézt cestu k nim. V druhé variantě je i pozice bomb utajena.

\subsubsection{Hráč prostředí}

Člověk si má myslet, že při hraní objevuje již předgenerované bludiště. Ve skutečnosti se bludiště tvoří postupně, jak ho hráč objevuje. Když hráč otevře nové dveře, vytvoří se mu nová chodba. Pro obě varianty hry platí, že rozmístění bomb je pevně dané, bomby nejsou umisťovány v průběhu hry. 

Druh dveří na každé možné konkrétní pozici je určen před začátkem hry. Z toho vyplývá, že hráč prostředí může barvu dveří ovlivňovat pouze nepřímo. Může ovlivnit umístění dveří, které následně určí jejich barvu.

Generování chodeb má jasně daná pravidla. Chodba může být libovolně dlouhá, i přes celý herní plán. Chodba může být zakončena zdí, nebo mohou na konci následovat další dveře. Z vygenerované chodby vedou vždy maximálně jedny dveře vlevo a jedny vpravo. Může se stát, že díky pozdějšímu napojení dvou chodem na sebe vznikne chodba s více dveřmi po levé, či pravé straně, ale nikdy nejsou na žádné straně vygenerovány dvě dveře naráz. Toto omezení je vynahrazeno možností ukončit chodbu dveřmi a ne zdí. Více dveří na jedné straně lze tak vygenerovat postupně pomocí dvou chodem stejným směrem.

I bez volby druhů dveří zbývá příliš mnoho možných tahů. Na mapě 40x40 polí může být délka generované chodby maximálně 39 polí dlouhá. Počet kombinací umístění dveří vlevo nebo vpravo se zakončením dveřmi, nebo zdí je $2*40^2$ (pozice 1-39, plus možnost nedat žádné dveře). Větvící faktor větší než 3200 by nebyl příliš použitelný pro herní algoritmy. 

Pro snížení větvícího faktoru nahradíme vždy několik tahů jedním. Provedeme abstrakci. HP neurčuje přesnou pozici dveří, ale jen přibližnou. Chodba je rozdělena do několika úseků a HP hráč si volí pouze, ve kterém úseku chce dveře. Konkrétní místo se vybere náhodně v rámci úseku. Dále HP nevybírá, jestli chodbu zakončí zdí, nebo dveřmi. Opět náhodná volba. Tímto byl větvící faktor HP hráče u bludišť o velikosti 40x40 snížen na maximální hodnotu 26.

Zásadní podmínkou pro generování bludiště bylo zakázat vygenerování neřešitelného bludiště. V žádném kroku nesmí být vygenerována chodba, která znemožní v budoucnu přístup k jakékoli z bomb. Bez důkazu bylo experimentálně ověřeno tvrzení, že pokud v jednom tahu existuje cesta mezi bombou a jedněmi z dveří přes zatím neodhalené čtverce mapy, tak v takovém případě existuje posloupnost tahů/možných chodeb, které skrz neodhalené čtverce vytvoří cestu mezi bombou a hráčem. Toto tvrzení by nebylo platné, kdyby nebyla povolena libovolná délka chodeb.

\subsubsection{Heuristika a uvěřitelnost}

Určení hodnosti hry hráčem závisí na variantě hry. Ve variantě s viditelnými bombami se spočítá manhattonská vzdálenost mezi hráčem a nejbližší bombou. Výsledná hodnost je dána rozdílem počtu kroků do konce hry a desetinásobkem spočítané vzdálenosti.

Ve druhé variantě hry nelze využít informací o poloze bomb. Heuristika navádí hráče k nejbližším dveřím s ohledem na blízkost dveří dalších. Jejím základem je počet kroků do konce mínus dvojnásobek reálné vzdálenosti k nejbližším dalším dveřím. K výsledku se přičte počet již nalezených bomb krát velká konstanta, aby hráč vždy upřednostnil odstranění bomby před hledáním další.

Pro jednoduchost se hráč ve hře pohybuje skokem mezi jednotlivými dveřmi a vždy se změří reálná vzdálenost takového pohybu. Z tohoto důvodu ve všech stavech mimo prvního hráč stojí na pozici nově otevřených dveří, nebo čerstvě odstraněné bomby. Z tohoto důvodu bonus/penalizace za otevření různých druhů dveří je už v heuristice započítán v proměnné počet kroků do konce hry.

Status funkce se v této hře výrazně liší od funkce hodnosti a je shodná pro obě varianty hry. Je složena se součtu vzdáleností k jednotlivým aktivním bombám($sumDist$) a počtu ještě neprozkoumaných čtverců ($undefinedTiles$) a samozřejmě počtu kroků do konce hry ($stepsToGameOver$). Výsledný vzorec : $status = stepsToGameOver - \frac{3}{2}sumDist-(undefinedTiles / 2 * coef)$. Koeficient coef je roven nule, jestliže hra skončila a hráč vyhrál, jinak je roven 1. 

Uvěřitelnost v této hře je definována na základě délek nově vytvářených chodeb. Cílem bude tvořit bludiště bez příliš dlouhých chodeb a s přibližně stejným zastoupením různých délek. V našem konkrétním případě se zaznamená počet chodeb do délky 2, délek 2/3, 4/5, 6/7, 8/9 a zvlášť počet chodem o délce 10 a delších. Výsledná uvěřitelnost hry je dána součtem stonásobku počtu chodeb delších než 9 a uvěřitelnosti četností prvních 5 délek.

\subsection{Ludo}

Ludo patří mezi zástupce stíhacích her. U nás je nejznámější zástupcem stíhacích her hra Člověče, nezlob se. Nejdřív uvedu společného rysy obou her. Znalci Člověče, nezlob se mohou zbytek odstavce přeskočit. Každý hráč má k dispozici 4 figury a jeho cílem je dovést všechny jeho figury z počáteční pozice do cíle. Herní plán se skládá ze 4 startovních a koncových pozic za každého hráče v jeho barvě a z 40 sdílených pozic uspořádaných do kružnice. Hráči se střídají ve svých tazích. Jeden tah se skládá z hodu kostkou a posunu jedné z figur hráče o počet pozic vpřed dle hodnoty na hozené kostce. Na žádném z polí nemohou být dvě figury na stejné pozici. Pokud by se tak mělo stát, figura nehrajícího hráče stojícího na stejné pozici jako nově posunutá figurka aktuálního hráče, je přesunuta zpět na startovní pozici. Ze startovní pozice na hlavní herní plán hráč může přesunout figuru, jestliže hodí na kostce šestku. Tuto šestku už nemůže použít pro pohyb jiné nebo stejné figury. Jestliže nemá hráč ani jednu z figur na hlavním plánu, může házet až třikrát, dokud nehodí šestku. Vítězí hráč, který jako první přesune všechny své figury do cílových pozic.

Hra Ludo do hry přidává bezpečné pozice. Některé pozice na herním plánu jsou odlišeny od ostatních. Jestliže na ní hráč má figuru, nemůže být odstraněna oponentem. Pokud oponent hodí kostkou přesně hodnotu, která by ho posunula na obsazenou bezpečnou pozici, tah nemůže provést. Další změnou je neexistence bonusových hodů po hození hodnoty 6, a to ani v případě, že pomocí 6 hráč figuru nově nasadil na herní plán.

V obou zmíněných variantách hráč má na začátku hry všechny figury na startovních pozicích v tzv. startovním domečku. Pro urychlení hry, ale i odstranění počáteční frustrace hráčů s menším štěstím, v implementované variantě hry všechny figury všech hráčů začínají na hlavním plánu.

\subsubsection{Hráč prostředí}

Hráč prostředí zde ovládá hrací kostku. Vzhledem k pravidlům hry je vliv HP hráče velice vysoký. Teoreticky je pouze na něm, který z ostatních hráčů vyhraje. Např. nemusí dovolit žádnému z hráčů hodit na kostce poslední potřebnou hodnotu pro umístění figury do cílové pozice. Z tohoto důvodu je zde zásadní metrikou uvěřitelnost.

\subsubsection{Použitá heuristika a uvěřitelnost}

\emph{Heuristiky} pro hodnotící a status funkci jsem použil totožné. Heuristika pro jednotlivé hráče vyjadřuje přibližně, kolik průměrně kol hráč potřebuje k dokončení hry. Hodnota se rovná součtu počtu kol potřebných k dosažení cíle všech figur hráče. 

Výpočet pro jednu figuru je následující. Průměrný hod má hodnotu 3,5. Počet kol k dosažení cíle je roven počtu polí před figurkou děleno 3,5. Pro nasazení figury je potřeba hodit 6. Abychom měli alespoň 50\% hození 6, musíme kostkou hodit 4 krát. ($1-\frac{5}{6}^4$) Uvažujme případ, kdy hráč nemá žádnou figuru na hracím plánu. V takovém případě hráč hází 3 krát, a tedy potřebuje $\frac{4}{3}$ tahů, které se přičtou k heuristice pro každou nenasazenou figuru. Figury před cílovým domečkem musí na kostce hodit hodnotu odpovídající volné pozici v cílovém domečku. Uvažujme pesimistický případ, kdy je vhodná pouze jedna hodnota. V takovém případě přičteme další 4 potřebné tahy k heuristice.

Jestliže figura nestojí na bezpečné pozici a zároveň za ní stojí do vzdálenosti 6 figura oponenta, přičteme k heuristice $\frac{1}{6}$ z potřebných tahů k dostání se do aktuální pozice figury.

\emph{Uvěřitelnost} hry závisí na hodech kostkou. Každý z hráčů očekává, že během hry každou hodí každou hodnotu kostky přibližně stejněkrát. Zároveň je nepravděpodobné, že by hráč po sobě hodil 3 a vícekrát stejnou hodnotu. Uvěřitelnost hráče se skládá ze dvou částí. Je součtem uvěřitelnosti četnosti hodů kostkou a penalizací za hod stejné hodnoty 3 a vícekrát. Výsledná uvěřitelnost je rovna sumě uvěřitelností pro jednotlivé hráče.

\subsection{Ztracená města}

Ztracená města patří mezi jednoduché karetní hry od společnosti Albi. Herní balíček se skládá ze 60 karet 5 barev představujících jednotlivé expedice. Každá expedice může být složena až ze 12 karet. První 3 karty jsou sázkové, zbylé mají hodnotu 2 až 10. Cílem hry je vytvářet co nejdelší a nejhodnotnější expedice a získat větší bodový zisk než soupeř. Na začátku hry si každý z hráčů lízne 8 karet ze zamíchaného balíčku. Hráči se postupně střídají v kolech. Každé kolo se skládá ze dvou částí - zahrání karty a dolíznutí karty. Při zahrání karty má hráč na výběr v přiložení karty do jeho existující expedice, nebo odložení karty na vrchol společného odkládacího balíčku příslušné barvy. Při přiložení karty do expedice musí být splněno pravidlo, že nově přidaná karta má vyšší hodnotu než je nejvyšší hodnota v dané expedici. Sázkové karty hráč může přidat do expedice pouze v případě, že v ní nemá již kartu nesázkovou. V druhé fázi si hráč dobere kartu do celkového počtu 8 karet. Má na výběr mezi zakrytým dobíracím balíčkem, nebo si může vzít kartu z vrchu jednoho odkládacího balíčku.

Bodové hodnocení expedic hráče se spočítá následovně. U každé expedice se sečtou hodnoty běžných karet. Z expedice se odečtou náklady v celkové hodnotě 20. V případě, že hráč má vyloženu 1, 2, nebo všechny tři sázkové karty, výslednou hodnotu vynásobí 2, 3, nebo 4. Expedice může mít ve výsledku zápornou hodnotu. Např. má-li hráč v expedici jednu sázkovou kartu a karty s hodnotami 4 a 5, bodové hodnocení se spočítá $(4 + 5 - 20) * 2$. Jestliže je expedice složena alespoň z 8 z celkových 12 karet, hráč si přičte dalších bonusových 20 bodů. (Tyto body se nenásobí)

Kompletní originální pravidla hry s vysvětlujícími obrázky lze nalézt na \cite{Neni}.

\subsubsection{Hráč prostředí}

Na začátku hry je náhodně rozdáno 8 karet pro každého hráče. Vždy, když se hráč rozhodne táhnout kartu z balíčku, kartu vybere hráč prostředí.

\subsubsection{Použitá heuristika a uvěřitelnost}

Podobně jako u hry Ludo, i zde je využívána stejná heuristika pro funkce status a hodnosti. Heuristika je vždy počítána jako by se jednalo o hru s úplnou informací. Můžeme ji počítat tímto způsobem, protože umělá inteligence nikdy nevolá heuristiku přímo na aktuální stav, ale vždy na stav ze stejného informačního setu, který stochasticky vygeneruje.

Heuristika odhaduje předpokládaný počet bodů získaných z jednotlivých expedic. Počítáme body pouze z expedic, které již byly založeny. Pro každou založenou expedici spočteme aktuální počet bodů a výsledek vynásobíme konstantou 10. Poté projdeme všechny karty v ruce, v balíčku a vrchní kartu z odložených karet, které ještě hráč může zahrát. (mají vyšší hodnotu než nejvyšší vyložená karta) Součet bodů těchto karet vynásobíme konstantami 8, 4, nebo 3 a 7 pro karty z ruky, balíčku a vrchní kartu. Pro karty z balíčku se vybere konstanta 3, jestliže hráč založí 4. expedici, jinak je 4. Heuristika je převzata z \cite{ch5LT} včetně koeficientů.

Aby hra byla \emph{uvěřitená}, jednotliví hráči musí během hry získat podobné počty barev a hodnot karet. Uvěřitelnost barvy je méně důležitá. Naopak pokud se pokusíme, aby každý hráč dostal od každé barvy během hry stejný počet karet, můžete to negativně ovlivnit průběh hry. Hodnoty karet rozdělíme do 4 kategorií - sázkové karty, hodnoty 2, 3, 4, hodnoty 5, 6, 7 a nakonec 8, 9, 10. Hra bude více uvěřitelná, když každý hráč během hry získá podobně karet z každé kategorie.

Výsledná uvěřitelnost je sumou uvěřitelností četnosti kategorií hodnot a poloviny uvěřitelnosti četností barev pro každého hráče zvlášť.

\section{Použité umělé inteligence}

\subsection{HillClimber}

Algoritmus HillClimber patří mezi základní jednoduché algoritmy. Ke své funkčnosti potřebuje funkci, jež dokáže získat všechny následující tahy a funkci, která je ohodnotí hodností. Ohodnocené tahy seřadí dle hodnosti a vybere nejlepší z nich. Algoritmus je určen pro hry Bludiště a Ludo.

Pro hru Ztracená města je určená varianta s IS (informační set). HillClimber IS před rozhodnutím provede několik desítek iterací. Na začátku iterace vygeneruje náhodně tah ze stejného informačního setu. Pro jednotlivé tahy si ukládáme sumu hodností přes všechny iterace. Můžeme si to dovolit, protože ve hře Ztracená města nejsou možné tahy hráče v jednom kole závislé na neúplné informaci, na soupeřově kartách v ruce. Výsledná sumární ohodnocení pro tahy seřadíme a opět vybereme nejlepší z nich.

\subsection{Max N}

Algoritmus Max N je rozšířenou variantou známějšího algoritmu minimax pro více hráčů a je zde použit pro hru Ludo. Algoritmus prochází herní strom do hloubky, která je zpravidla omezena konstantou. V dané hloubce ohodnotí stav pomocí funkce hodnosti pro všechny hráče a výsledný soubor hodností propaguje výše. 

Každý vnitřní uzel odpovídá tahu jednoho hráče. Ve chvíli, kdy se sesbírají soubory hodností ze všech potomků, se další postup liší dle typu hráče. Jedná-li se o hráče prostředí, tak jde z hlediska AI o tzv. chance node. V tomto případě se o patro výše propaguje zprůměrovaná skupina hodností z potomků. (uvažujeme hry, kde jednotlivé tahy mají stejnou pravděpodobnost)

Běžný hráč vybere z potomků takovou skupinu hodností, kde hodnost tohoto hráče je co nejvyšší a tuto skupinu opět propaguje o patro výš.

ALGORITMUS : DOPLNIT TODO

\subsection{Monte Carlo}

\subsection{Minimax IS}

\subsection{Škálování}

Všechny uvedené algoritmy umožňují škálování umělé inteligence. Škálování potřebuje pro simulaci různě silných hráčů. Vedle algoritmu můžeme nastavit parametr úrovně mezi 1 a 100. Použil jsem dva způsoby škálování. 

První je využit pro HillClimber včetně varianty IS a pro Max N. Ve všech případech algoritmus nevrátí jen nejlepší možný tah, ale všechny tahy seřazené dle jejich vhodnosti. Ze seřazených tahů se nevybírá nejlepší tah, ale nejpravděpodobněji se vybere ten, který se nachází v $\frac{úroveň}{100}$ části pole. Vybere se tah dle gaussovského rozdělení pravděpodobnosti se střední hodnotou v $\frac{úroveň}{100}$ a odchylkou $0,4$. Kdybychom vybírali tah přesně v dané části pole, ztratili bychom výraznou část tohoto škálování. Např. u hry Člověče, nezlob se, kde má hráč v jednu chvíli na výběr maximálně 4 tahy, by se hráči s úrovněmi 76-100 chovali zcela stejně.

Druhý typ škálování je využit u algoritmů Monte Carlo a Minimax IS. Úroveň zde ovlivňuje počet iterací algoritmu. S vyšší úrovní proběhne větší počet iterací, a tím je pravděpodobnější inteligentnější chování.

\endinput
%%
%% End of file `ch01.tex'.
