\chapter{Testující prostředí}



\section{Použité hry}

Pro testování navržených algoritmů v předchozí kapitole jsem si zvolil následující tři relativně jednoduché hry - Bludiště, Ludo a Ztracená města. Hry se liší počtem hráčů, úplností informací a determinismus. Zařazení do jednotlivých kategorií znázorňuje následující tabulka \ref{tab:klasifikaceher}.

\begin{table*}[b]\footnotesize
\vspace*{0mm}
\caption{{\label{tab:klasifikaceher}} Klasifikace her do různých tříd dle počtu hráčů, determinismu a úplnosti informace. }
\vspace*{0mm}
\label{shadowtable}
\begin{center}
\begin{tabular}{| l || c | c | c |}
\hline
Název hry & Počet hráčů & Determinismus & Informace \\
\hline
\hline
Bludiště & 1 & Bez náhody (z pohledu hráče) & Neúplná \\ \hline
Ludo & 4 & S náhodou & Úplná \\ \hline
Ztracená města & 2 & S náhodou & Neúplná \\ \hline
\end{tabular}
\end{center}
\end{table*}

Schéma podsekcí jednotlivých her je stejné. Nejdříve popíši cíl a pravidla hry. Následuje popis hráče prostředí (HP) v dané hře. Nakonec popisuji funkci hodnosti, status funkci a výpočet uvěřitelnosti, které jsou využívány metrikami pro měření zábavnosti. Pojmy funkcí hodnosti a statusu vychází z algoritmu Dynamická úroveň (\ref{sec-dynlevel}).

\subsection{Bludiště}

Bludiště je jednoduchou hrou pro jednoho hráče. Terorista v bludišti umístil několik bomb. Cílem hráče je tyto bomby včas najít a zneškodnit. Hráč má k dispozici plánek s umístěním všech bomb, ale bohužel jsou v plánku vyznačeny i neexistující bomby. Hráč musí zkontrolovat všechny místa označená na mapě. Buď zjistí, že dané místo není dostupné, a tudíž se tam bomba nenachází, nebo místo dostupné je a bombu musí zneškodnit. (stačí dojít na její místo) 

Herní plán bludiště se skládá z 2D čtvercové sítě. Jednotlivé čtverce mapy představují zdi, dveře, bomby, nebo průchozí pole. Hráč vidí bludiště z ptačí perspektivy a bludiště objevuje postupně. Vždy po otevření dveří se mu zobrazí chodba za nimi. Všechny bomby vybuchnou ve stejný čas. Čas je zde měřen na počet kroků hráče.

Terorista navíc na některé dveře připevnil senzory, a poté dveře přemaloval namodro, nebo červeně. Jestliže hráč otevře modré dveře, získá čas navíc. Otevře-li červené, čas se mu zkrátí.

\subsubsection{Hráč prostředí}

Člověk si má myslet, že při hraní objevuje již dříve vygenerované bludiště. Ve skutečnosti se bludiště tvoří postupně, jak ho hráč objevuje. Když hráč otevře nové dveře, vytvoří se mu nová chodba. 

Druh dveří na každé možné konkrétní pozici je určen před začátkem hry. Z toho vyplývá, že hráč prostředí může barvu dveří ovlivňovat pouze nepřímo. Může ovlivnit umístění dveří, které následně určí jejich barvu.

Generování chodeb má jasně daná pravidla. Chodba může být libovolně dlouhá, i přes celý herní plán. Chodba může být zakončena zdí, nebo mohou na konci následovat další dveře. Z vygenerované chodby vedou vždy maximálně jedny dveře vlevo a jedny vpravo. Může se stát, že díky pozdějšímu napojení dvou chodem na sebe vznikne chodba s více dveřmi po levé, či pravé straně, ale nikdy nejsou na žádné straně vygenerovány dvě dveře naráz. Toto omezení je vynahrazeno možností ukončit chodbu dveřmi a ne zdí. Více dveří na jedné straně lze tak vygenerovat postupně pomocí dvou chodeb vytvořených ve stejném směru.

I bez volby druhů dveří zbývá příliš mnoho možných tahů. Na mapě 40x40 polí může být délka generované chodby maximálně 39 polí dlouhá. Počet kombinací umístění dveří vlevo nebo vpravo se zakončením dveřmi, nebo zdí je $2*40^2$ (pozice 1-39, plus možnost nedat žádné dveře). Větvící faktor větší než 3200 by nebyl příliš použitelný pro herní algoritmy. 

Pro snížení větvícího faktoru nahradíme vždy několik tahů jedním. Provedeme abstrakci. HP neurčuje přesnou pozici dveří, ale jen přibližnou. Chodba je rozdělena do několika úseků a HP hráč si volí pouze, ve kterém úseku chce dveře. Konkrétní místo se vybere náhodně v rámci úseku. Dále HP nevybírá, jestli chodbu zakončí zdí, nebo dveřmi. Opět náhodná volba. Tímto byl větvící faktor HP hráče u bludišť o velikosti 40x40 snížen na maximální hodnotu 26.

\subsubsection{Heuristika a uvěřitelnost}

Heuristika pro hodnost se spočítá jako manhattonská vzdálenost mezi hráčem a nejbližší bombou. Výsledná hodnost je dána rozdílem počtu kroků do konce hry a desetinásobkem spočítané vzdálenosti.

Pro jednoduchost se hráč ve hře pohybuje skokem mezi jednotlivými dveřmi a vždy se změří reálná vzdálenost takového pohybu. Z tohoto důvodu ve všech stavech mimo prvního hráč stojí na pozici nově otevřených dveří, nebo čerstvě odstraněné bomby. Z tohoto důvodu bonus/penalizace za otevření různých druhů dveří je už v heuristice započítán v proměnné počet kroků do konce hry.

Status funkce se v této hře výrazně liší od funkce hodnosti. Je složena se součtu vzdáleností k jednotlivým aktivním bombám($sumDist$) a počtu ještě neprozkoumaných čtverců ($undefinedTiles$) a samozřejmě počtu kroků do konce hry ($stepsToGameOver$). Výsledný vzorec : $status = stepsToGameOver - \frac{3}{2}sumDist-(undefinedTiles / 2 * coef)$. Koeficient coef je roven nule, jestliže hra skončila a hráč vyhrál, jinak je roven 1. 

Uvěřitelnost v této hře je definována na základě délek nově vytvářených chodeb. Cílem bude tvořit bludiště bez příliš dlouhých chodeb a s přibližně stejným zastoupením různých délek. V našem konkrétním případě se zaznamená počet chodeb do délky 2, délek 2/3, 4/5, 6/7, 8/9 a zvlášť počet chodem o délce 10 a delších. Výsledná uvěřitelnost hry je dána součtem tisícinásobku počtu chodeb delších než 9 a uvěřitelnosti četností pole prvních 5 délek.

\subsection{Ludo}

Ludo patří mezi zástupce stíhacích her. U nás je nejznámější zástupcem stíhacích her hra Člověče, nezlob se. Nejdřív uvedu společného rysy obou her. Znalci Člověče, nezlob se mohou zbytek odstavce přeskočit. Každý hráč má k dispozici 4 figury a jeho cílem je dovést všechny jeho figury z počáteční pozice do cíle. Herní plán se skládá ze 4 startovních a koncových pozic za každého hráče v jeho barvě a z 40 sdílených pozic uspořádaných do kružnice. Hráči se střídají ve svých tazích. Jeden tah se skládá z hodu kostkou a posunu jedné z figur hráče o počet pozic vpřed dle hodnoty na hozené kostce. Na žádném z polí nemohou být dvě figury na stejné pozici. Pokud by se tak mělo stát, figura nehrajícího hráče stojícího na stejné pozici jako nově posunutá figurka aktuálního hráče, je přesunuta zpět na startovní pozici. Ze startovní pozice na hlavní herní plán hráč může přesunout figuru, jestliže hodí na kostce šestku. Tuto šestku už nemůže použít pro pohyb jiné nebo stejné figury. Jestliže nemá hráč ani jednu z figur na hlavním plánu, může házet až třikrát, dokud nehodí šestku. Vítězí hráč, který jako první přesune všechny své figury do cílových pozic.

Hra Ludo do hry přidává bezpečné pozice. Některé pozice na herním plánu jsou odlišeny od ostatních. Jestliže na ní hráč má figuru, nemůže být odstraněna oponentem. Pokud oponent hodí kostkou přesně hodnotu, která by ho posunula na obsazenou bezpečnou pozici, tah nemůže provést. Další změnou je neexistence bonusových hodů po hození hodnoty 6, a to ani v případě, že pomocí 6 hráč figuru nově nasadil na herní plán.

V obou zmíněných variantách hráč má na začátku hry všechny figury na startovních pozicích v tzv. startovním domečku. Pro urychlení hry, ale i odstranění počáteční frustrace hráčů s menším štěstím, v implementované variantě hry všechny figury všech hráčů začínají na hlavním plánu.

\subsubsection{Hráč prostředí}

Hráč prostředí zde ovládá hrací kostku. Vzhledem k pravidlům hry je vliv HP hráče velice vysoký. Teoreticky je pouze na něm, který z ostatních hráčů vyhraje. Např. nemusí dovolit žádnému z hráčů hodit na kostce poslední potřebnou hodnotu pro umístění figury do cílové pozice. Z tohoto důvodu je zde zásadní metrikou uvěřitelnost.

\subsubsection{Použitá heuristika a uvěřitelnost}

\emph{Heuristiky} pro hodnotící a status funkci jsem použil totožné. Heuristika pro jednotlivé hráče vyjadřuje přibližně, kolik průměrně kol hráč potřebuje k dokončení hry. Hodnota se rovná součtu počtu kol potřebných k dosažení cíle všech figur hráče. 

Výpočet pro jednu figuru je následující. Průměrný hod má hodnotu 3,5. Počet kol k dosažení cíle je roven počtu polí před figurkou děleno 3,5. Pro nasazení figury je potřeba hodit 6. Abychom měli alespoň 50\% hození 6, musíme kostkou hodit 4 krát. ($1-\frac{5}{6}^4$) Uvažujme případ, kdy hráč nemá žádnou figuru na hracím plánu. V takovém případě hráč hází 3 krát, a tedy potřebuje $\frac{4}{3}$ tahů, které se přičtou k heuristice pro každou nenasazenou figuru. Figury před cílovým domečkem musí na kostce hodit hodnotu odpovídající volné pozici v cílovém domečku. Uvažujme pesimistický případ, kdy je vhodná pouze jedna hodnota. V takovém případě přičteme další 4 potřebné tahy k heuristice.

Jestliže figura nestojí na bezpečné pozici a zároveň za ní stojí do vzdálenosti 6 figura oponenta, přičteme k heuristice $\frac{1}{6}$ z potřebných tahů k dostání se do aktuální pozice figury.

\emph{Uvěřitelnost} hry závisí na hodech kostkou. Každý z hráčů očekává, že během hry každou hodí každou hodnotu kostky přibližně stejněkrát. Zároveň je nepravděpodobné, že by hráč po sobě hodil 3 a vícekrát stejnou hodnotu. Uvěřitelnost hráče se skládá ze dvou částí. Je součtem uvěřitelnosti četnosti hodů kostkou a penalizací za hod stejné hodnoty 3 a vícekrát. Výsledná uvěřitelnost je rovna sumě uvěřitelností pro jednotlivé hráče.

\subsection{Ztracená města}

Ztracená města patří mezi jednoduché karetní hry od společnosti Albi. Herní balíček se skládá ze 60 karet 5 barev představujících jednotlivé expedice. Každá expedice může být složena až ze 12 karet. První 3 karty jsou sázkové, zbylé mají hodnotu 2 až 10. Cílem hry je vytvářet co nejdelší a nejhodnotnější expedice a získat větší bodový zisk než soupeř. Na začátku hry si každý z hráčů lízne 8 karet ze zamíchaného balíčku. Hráči se postupně střídají v kolech. Každé kolo se skládá ze dvou částí - zahrání karty a dolíznutí karty. Při zahrání karty má hráč na výběr v přiložení karty do jeho existující expedice, nebo odložení karty na vrchol společného odkládacího balíčku příslušné barvy. Při přiložení karty do expedice musí být splněno pravidlo, že nově přidaná karta má vyšší hodnotu než je nejvyšší hodnota v dané expedici. Sázkové karty hráč může přidat do expedice pouze v případě, že v ní nemá již kartu nesázkovou. V druhé fázi si hráč dobere kartu do celkového počtu 8 karet. Má na výběr mezi zakrytým dobíracím balíčkem, nebo si může vzít kartu z vrchu jednoho odkládacího balíčku.

Bodové hodnocení expedic hráče se spočítá následovně. U každé expedice se sečtou hodnoty běžných karet. Z expedice se odečtou náklady v celkové hodnotě 20. V případě, že hráč má vyloženu 1, 2, nebo všechny tři sázkové karty, výslednou hodnotu vynásobí 2, 3, nebo 4. Expedice může mít ve výsledku zápornou hodnotu. Např. má-li hráč v expedici jednu sázkovou kartu a karty s hodnotami 4 a 5, bodové hodnocení se spočítá $(4 + 5 - 20) * 2$. Jestliže je expedice složena alespoň z 8 z celkových 12 karet, hráč si přičte dalších bonusových 20 bodů. (Tyto body se nenásobí)

Kompletní originální pravidla hry s vysvětlujícími obrázky lze nalézt na \cite{Neni}.

\subsubsection{Hráč prostředí}

Na začátku hry je náhodně rozdáno 8 karet pro každého hráče. Vždy, když se hráč rozhodne táhnout kartu z balíčku, kartu vybere hráč prostředí.

\subsubsection{Použitá heuristika a uvěřitelnost}

Podobně jako u hry Ludo, i zde je využívána stejná heuristika pro funkce status a hodnosti. Heuristika je vždy počítána jako by se jednalo o hru s úplnou informací. Můžeme ji počítat tímto způsobem, protože umělá inteligence nikdy nevolá heuristiku přímo na aktuální stav, ale vždy na stav ze stejného informačního setu, který stochasticky vygeneruje.

Heuristika odhaduje předpokládaný počet bodů získaných z jednotlivých expedic. Počítáme body pouze z expedic, které již byly založeny. Pro každou založenou expedici spočteme aktuální počet bodů a výsledek vynásobíme konstantou 10. Poté projdeme všechny karty v ruce, v balíčku a vrchní kartu z odložených karet, které ještě hráč může zahrát. (mají vyšší hodnotu než nejvyšší vyložená karta) Součet bodů těchto karet vynásobíme konstantami 8, 4, nebo 3 a 7 pro karty z ruky, balíčku a vrchní kartu. Pro karty z balíčku se vybere konstanta 2, jestliže hráč založí 4. expedici, jinak je 3. Heuristika je převzata z \cite{ch5LT} včetně koeficientů.

Aby hra byla \emph{uvěřitená}, jednotliví hráči musí během hry získat podobné počty barev a hodnot karet. Uvěřitelnost barvy je méně důležitá. Naopak pokud se pokusíme, aby každý hráč dostal od každé barvy během hry stejný počet karet, můžete to negativně ovlivnit průběh hry. Hodnoty karet rozdělíme do 4 kategorií - sázkové karty, hodnoty 2, 3, 4, hodnoty 5, 6, 7 a nakonec 8, 9, 10. Hra bude více uvěřitelná, když každý hráč během hry získá podobně karet z každé kategorie.

Výsledná uvěřitelnost je sumou uvěřitelností četnosti kategorií hodnot a poloviny uvěřitelnosti četností barev pro každého hráče zvlášť.

\section{Použité umělé inteligence}

V aplikaci jsou implementovány dva druhy hráčů. První druh představuje hráče prostředí, kteří se starají o náhodný prvek ve hře. Druhým typem jsou běžní hráči, kteří se chovají relativně rozumně.

Pro všechny tři hry jsem připravil hráče založené na DDA algoritmech Dynamická úroveň a POSM uvedených ve třetí kapitole. Tito hráči slouží pro porovnání mého nového přístupu s již existujícími.

\subsection{Hráči prostředí}

Prvním hráčem prostředí je hráč hrající náhodně. S tímto hráčem se všechny hry chovají zcela přirozeně jako by v nich tento typ hráče vůbec nebyl. Tento hráč byl přidán pro srovnání naměřených metrik u her, které neovlivňuje hráč prostředí, a které naopak ovlivňuje.

Dále je v aplikaci implementována trojice hráčů prostředí navržených v předcházející kapitole.

Hra Bludiště disponuje dvěma hráči navíc. Obsahuje hráče založené na algoritmech POSM a Dynamická úroveň. Tito hráči dělají ze hry Bludiště hru se dvěma soupeřícími hráči. Klasický hráč má za úkol splnit úkol, nalézt všechny bomby a jeho soupeř se mu v tom snaží zabránit. 

\subsection{Běžní hráči}

V testovacím prostředí existuje několik hráčů postavených na různých algoritmech. Aplikace byla navržena tak, aby každý hráč byl použitelný pro každou hru. Některé kombinace her a hráčů nebyly příliš rozumné, a proto v aplikaci jsou povoleny pouze některé.

Ve všech hrách lze využít hráče hrajícího náhodně. Slouží pouze jako jeden z benchmarků pro ostatní algoritmy. Pomohl např. objevit chybu v dřívější heuristice pro Člověče, nezlob se. Náhodný hráč vyhrával častěji než hráč využívající heuristiku.

U her Bludiště a Ludo je aktivní další jednoduchý hráč nazvaný HillClimber. Ke své funkčnosti potřebuje funkci, jež dokáže získat všechny následující tahy a funkci, která je ohodnotí hodností. Ohodnocené tahy seřadí dle hodnosti a vybere nejlepší z nich.

Pro hru Ztracená města je určená varianta s IS (informační set). HillClimber IS před rozhodnutím provede několik desítek iterací. Na začátku iterace vygeneruje náhodně stav hry ze stejného informačního setu jako je aktuální stav. Nad tímto stavem provede základní algoritmus Hill Climber, ohodnotí všechny tahy z něj. Pro jednotlivé tahy si ukládáme sumu hodností přes všechny iterace. Můžeme si to dovolit, protože ve hře Ztracená města nejsou možné tahy hráče v jednom kole závislé na neúplné informaci, na soupeřově kartách v ruce. Výsledná sumární ohodnocení pro tahy seřadíme a opět vybereme nejlepší z nich.

Bludiště žádné jiné hráči nevyužívá, jelikož hráči nemají být seznámeni se způsobem tvorby bludiště.

Ve hře Ludo nalezneme hráče založené na Monte Carlo prohledávání stromu a na vícehráčové variantě algoritmu Minimax, na algoritmu Max N. Oba algoritmy byly stručně popsány na začátku předchozí kapitoly.

Pro hru Ztracená města jsem se snažil vytvořit hráče, který nebude podvádět (nebude znát karty v soupeřově ruce a ani skryté v balíčku) a zároveň bude lepší než jednoduchý algoritmus HillClimberIS.
Nejdříve jsem se pokoušel o různé varianty hráče založené na Monte Carlo prohledávání stromu. Bohužel všechny varianty hráče se chovali v průměru hůře než jednoduchý HillClimber IS. Z tohoto důvodu jsem se rozhodl pro implementaci hráče navrženého Pedrem Vasconcelosem \cite{ch5LT} pro stejnou hru a nazval jsem ho Minimax IS.

Hry Ludo a Ztracená města navíc obsahují dva hráče implementující DDA mechanismus. Jsou to hráči Dynamická úroveň a POSM pojmenovaní dle algoritmů popsaných ve třetí kapitole.

\subsection{Minimax IS}

Algoritmus Minimax IS se od klasického Minimaxu liší obdobně jako HillClimber IS od HillClimberu. 

Algoritmus v iteracích provádí dva kroky. Nejdříve vytvoří na základě aktuálního stavu stav ze stejné informační množiny. V našem případě se jedná o zamíchání neznámých soupeřových karet s balíčkem a z takto zamíchaných karet se doplní soupeřova ruka. Nad novým stavem se zavolá Minimax s prořezáváním alfa, beta. Zapamatuje se nejlepších tah dle Minimaxu a pokračuje se další iterací. Nakonec hráč zahraje tah, který byl nejvíce krát vybrán Minimaxem.

Vasconcelos herní strom prořezává do hloubky i do šířky. Prořezávání do hloubky je dle očekávání. V určité hloubce stromu se stav ohodnotí dle heuristické funkce. Při prořezávání do šířky se z každého uzlu ohodnotí všechny následující tahy dle stejné heuristické funkce, Minimax prochází pouze tři nejlepší z nich.

Hloubku stromu a počet iterací mění v průběhu hry. Čím méně karet zbývá v balíčku, tím provádí méně iterací, ale naopak zvedá maximální hloubku stromu.

Oproti originální verzi algoritmu jsem provedl několik změn. Každou změnu jsem vždy otestoval vůči hráčům HillClimber IS i proti HillClimber, který pracuje s úplnou informací. Ukázalo se, že šířka stromu 3 hned od první hloubky je příliš málo. Z tohoto důvodu v hloubce 1 ořezávám strom na šířku 10 a až poté ořezávám na šířku 3. Dále jsem upravil poměr počtu iterací a hloubky stromu. Čas získaný snížením maximální hloubky stromu šlo využít pro zvětšení počtu iterací. Tato změna také přispěla k lepšímu chování. 

Poslední, ale ne nedůležitou změnou, je úprava generování stavu ze stejné informační množiny. Každý hráč si pro všechny karty, které doposud neviděl, udržuje hodnotu pravděpodobnosti, s jakou je daná karta v soupeřově ruce. Pravděpodobnosti se upravují v průběhu hry na základě soupeřových tahů dle několika pravidel získaných od znalce hry. Příkladem je pravidlo, které se použije, pokud soupeř vyloží novou expedici. V takovém případě se zvedne pravděpodobnost pro karty stejné barvy a vyšší hodnoty než je vyložená karta. Naopak lze předpokládat, že soupeř nevlastní žádnou kartu nižší hodnoty stejné barvy, protože tah vyložení takové karty by dominoval provedenému tahu. Stav ze stejné informační množiny se generuje na základě těchto pravděpodobností a ne dle uniformního rozdělení pravděpodobnosti jako tomu bylo v původním algoritmu.

\subsection{Škálování}

Všechny uvedené algoritmy umožňují škálování umělé inteligence. Škálování potřebuje pro simulaci různě silných hráčů. Vedle algoritmu můžeme nastavit parametr úrovně mezi 1 a 100.

Každý z algoritmů pro běžné hráče je upraven tak, aby nevracel pouze nejlepší možný tah, ale aby vrátil ohodnocené všechny tahy. Ohodnocené tahy se seřadí vzestupně.
Ze seřazených tahů se nevybírá nejlepší tah, ale nejpravděpodobněji se vybere ten, který se nachází v $\frac{úroveň}{100}$. Vybere se tah dle gaussovského rozdělení pravděpodobnosti se střední hodnotou v $\frac{úroveň}{100}$ a odchylkou $0,4$. Kdybychom vybírali tah přesně v daném zlomku, ztratili bychom výraznou část tohoto škálování. Např. u hry Člověče, nezlob se, kde má hráč v jednu chvíli na výběr maximálně 4 tahy, by se hráči s úrovněmi 76-100 chovali zcela stejně.

\endinput
%%
%% End of file `ch01.tex'.
